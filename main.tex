%% The first command in your LaTeX source must be the \documentclass command.
%%%% Small single column format, used for CIE, CSUR, DTRAP, JACM, JDIQ, JEA, JERIC, JETC, PACMCGIT, TAAS, TACCESS, TACO, TALG, TALLIP (formerly TALIP), TCPS, TDSCI, TEAC, TECS, TELO, THRI, TIIS, TIOT, TISSEC, TIST, TKDD, TMIS, TOCE, TOCHI, TOCL, TOCS, TOCT, TODAES, TODS, TOIS, TOIT, TOMACS, TOMM (formerly TOMCCAP), TOMPECS, TOMS, TOPC, TOPLAS, TOPS, TOS, TOSEM, TOSN, TQC, TRETS, TSAS, TSC, TSLP, TWEB.
% \documentclass[acmsmall]{acmart}

%%%% Large single column format, used for IMWUT, JOCCH, PACMPL, POMACS, TAP, PACMHCI
% \documentclass[acmlarge,screen]{acmart}

%%%% Large double column format, used for TOG
% \documentclass[acmtog, authorversion]{acmart}

%%%% Generic manuscript mode, required for submission
%%%% and peer review
\documentclass[manuscript, anonymous, review]{acmart}
\usepackage{todonotes}
\usepackage{xcolor}

%% Fonts used in the template cannot be substituted; margin 
%% adjustments are not allowed.
%%
%% \BibTeX command to typeset BibTeX logo in the docs
\AtBeginDocument{%
  \providecommand\BibTeX{{%
    \normalfont B\kern-0.5em{\scshape i\kern-0.25em b}\kern-0.8em\TeX}}}

%% Rights management information.  This information is sent to you
%% when you complete the rights form.  These commands have SAMPLE
%% values in them; it is your responsibility as an author to replace
%% the commands and values with those provided to you when you
%% complete the rights form.
\setcopyright{acmcopyright}
\copyrightyear{2018}
\acmYear{2018}
\acmDOI{XXXXXXX.XXXXXXX}

%% These commands are for a PROCEEDINGS abstract or paper.
\acmConference[Conference acronym 'XX]{Make sure to enter the correct conference title from your rights confirmation emai}{June 03--05,
  2018}{Woodstock, NY}


\acmBooktitle{Woodstock '18: ACM Symposium on Neural Gaze Detection, June 03--05, 2018, Woodstock, NY} 
\acmPrice{15.00}
\acmISBN{978-1-4503-XXXX-X/18/06}


%%
%% Submission ID.
%% Use this when submitting an article to a sponsored event. You'll
%% receive a unique submission ID from the organizers
%% of the event, and this ID should be used as the parameter to this command.
%%\acmSubmissionID{123-A56-BU3}

%%
%% end of the preamble, start of the body of the document source.
\begin{document}

%%
%% The "title" command has an optional parameter,
%% allowing the author to define a "short title" to be used in page headers.
\title{Empathic Smart Hybrid Buildings}

%%
%% The "author" command and its associated commands are used to define
%% the authors and their affiliations.
%% Of note is the shared affiliation of the first two authors, and the
%% "authornote" and "authornotemark" commands
%% used to denote shared contribution to the research.
\author{Shruti Rao}
\email{s.rao@uva.nl}
\orcid{1234-5678-9012}
\affiliation{
  \institution{University of Amsterdam}
  \city{Amsterdam}
  \country{The Netherlands}
}



%%
%% By default, the full list of authors will be used in the page
%% headers. Often, this list is too long, and will overlap
%% other information printed in the page headers. This command allows
%% the author to define a more concise list
%% of authors' names for this purpose.
\renewcommand{\shortauthors}{Rao et al.}

%%
%% The abstract is a short summary of the work to be presented in the
%% article.
\begin{abstract}
%   Abstracts should be about 150 words.
Given that people spend a significant amount of time within ``smart built spaces", designing spaces considering the impact that it may have on occupants’ comfort and emotions is a challenge for Human-Building Interaction (HBI).

In this position paper, we offer the view of designing smart buildings through the lens of material experiences design, whereby materials (both tangible and intangible) may impact occupants experiences of comfort and emotions.

To that end, we describe our case study and how our expected findings may aid in identification of novel, subjective human experiences (sensory, affective, interpretive and performative) with materials in hybrid spaces.

These cues may serve as a preliminary step towards designing comfort-enabling materials  and experiences in a smart, hybrid learning environment. 
\end{abstract}

%%
%% The code below is generated by the tool at http://dl.acm.org/ccs.cfm.
%% Please copy and paste the code instead of the example below.
%%

\begin{CCSXML}
<ccs2012>
   <concept>
       <concept_id>10003120</concept_id>
       <concept_desc>Human-centered computing</concept_desc>
       <concept_significance>500</concept_significance>
       </concept>
   <concept>
       <concept_id>10003120.10003123</concept_id>
       <concept_desc>Human-centered computing~Interaction design</concept_desc>
       <concept_significance>500</concept_significance>
       </concept>
 </ccs2012>
\end{CCSXML}

\ccsdesc[500]{Human-centered computing}
\ccsdesc[500]{Human-centered computing~Interaction design}

%%
%% Keywords. The author(s) should pick words that accurately describe
%% the work being presented. Separate the keywords with commas.
\keywords{affective computing, comfort, smart built environments, hybrid learning spaces, empathic design}

%% A "teaser" image appears between the author and affiliation
%% information and the body of the document, and typically spans the
%% page.
% \begin{teaserfigure}
%   \includegraphics[width=\textwidth]{sampleteaser}
%   \caption{Seattle Mariners at Spring Training, 2010.}
%   \Description{Enjoying the baseball game from the third-base
%   seats. Ichiro Suzuki preparing to bat.}
%   \label{fig:teaser}
% \end{teaserfigure}

 
%%
%% This command processes the author and affiliation and title
%% information and builds the first part of the formatted document.
\maketitle

\section{Introduction}

% Emotions in humans
Emotions serve as an implicit information communication channel between humans. Much of human communication takes place in the form of emotional exchanges that are an implicit channel for conveying information. The emotions expressed can be in the form of voice and speech, facial expressions, and behavioural patterns. These cues are recognised by humans who infer and react appropriately.

% Human-Buildings Link
In contrast to human interactions, we spend a significant amount of time inside increasingly digital buildings that creates a new dimension of relatively under-explored interaction. Information and communication technologies (ICT) and building energy management systems (BEMS) have resulted in buildings transforming from inanimate structures into interactive and communicative objects \cite{nembrini2017human}. And although buildings of the digital age can intelligently change the environmental variables (temperature, light, air quality purification) to be energy efficient and enable occupants comfort, emotion-sensing and empathic behaviour remain under-explored. 

% position
In this position paper, we consider ``smart built environments" that combine architecture with intelligent artefacts. These buildings enable for a new way of sensory perception of spaces by humans and therefore human-building interactions. We posit the need for empathic buildings that can infer occupants affective states and react appropriately to ensure occupants well being and comfort within buildings. 

% paper outline
The remainder of this paper provides a background on architecture in the digital age and the move towards human-centric buildings. We then describe our case study as a preliminary step towards understanding the perception of empathy by occupants of a smart building, and the contributions our findings can make towards empathc smart buildings.

% % comfort
% A key aspect of human experience within built environments (and by extension smart environments) is derived from the physical and emotional comfort of the occupants \cite{alavi2017comfort}. This subjective comfort which is largely as a result of qualities of the built environment can play a role in impacting the occupants awareness and behaviour within that environment.  




\section{Background}
\todo{Some more info to link empathic and comfort design?}
% HBI
Human-building interaction (HBI) is a burgeoning area of research that focuses on capturing, understanding and enhancing human interactions and experiences both with and within ``smart built environments" \cite{alavi2016future}. The primary goal of HBI being a framework that can be used to understand, compare, and converge research efforts from the fields of HCI, design, and architecture in envisioning and shaping the future of living spaces and all that they encompass \cite{nembrini2017human, alavi2018artifacts}. 

% Comfort and Affect in Built Environment
The concept of comfort is central to occupants within built environments. Comfort is understood as occupants' physiological and affective (emotional) responses to the built environment \cite{alavi2017comfort}. HBI especially examines the relationship between occupant comfort and four physical characteristics of the indoor environment - temperature, air, light, and sound \cite{hawkes2007environmental, bluyssen2009indoor}. 

% Examples of existing work
Much of the work in comfort studies focus on designing ``optimally informative" systems such as temperature calendars, air quality forecasts, noise level indicators, and wearables that appropriately inform occupants of their environment and also allow them to engage with comfort parameters to certain degrees  \cite{costanza2016bit, milenkovic2013improving, kim2020designing}. Other works focus on a ``gamified" approach to engage users with their environment and activities in a socially inclusive manner \cite{mathur2015tiny, kwallek1997impact, zhong2022augmenting}. 

% Our perspective
Therefore, to our knowledge comfort studies tend to focus on designing novel interfaces and digital systems for physical comfort of occupants. Instead, in this position paper we wish to reconsider designing for occupants' affective comfort through the lens of empathic design principles in a smart hybrid learning environment. 


\section{Empathic Buildings Design For Comfort in Hybrid Learning Spaces}

% Why should we design keeping in mind materiality
A significant consequence of smart buildings is that occupants find themselves physically immersed within an interactive object, and therefore experience interactions in a multi-sensory manner \cite{nembrini2017human}. Moreover, providing for subjective comfort to occupants within a hybrid space that is used by different people with varying usage goals remains a relevant question. 

We believe that the materials experience framework can be used as a novel means to understand the relationship between materials and occupants at different experiential levels (sensorial, affective, interpretive and performative) within hybrid spaces \cite{giaccardi2015foundations}. Different tangible and intangible materials in hybrid spaces can be utilised to shape the ways in which occupants experience emotional and physical comfort, and consequently learning attitudes. Moreover, material
experiences can play a more ubiquitous role in fostering occupants' interactions with non-digital elements of the building through digital means - a key concern in HBI \cite{nembrini2017human}. 

% Our proposed case study
\subsection{Our Proposed Case Study} 
Towards developing empathic architecture that can infer occupant emotions and comfort, we are undertaking a preliminary case study in the aforementioned building. We aim to identify occupants' novel and subjective experiences that arise from interactions within the smart building. We are particularly interested in understanding experiences of occupant comfort (environmental) and emotions \cite{alavi2017comfort} which is key in built environments.  We wish to thereafter identify specific influential factors (tangible and intangible) that determine the overall sense of comfort and emotions in smart buildings spaces. The case study comprises two phases: (a) emotion and comfort label collection, and (b) building walk. First, we will conduct a short survey to obtain a large number of occupant emotion and comfort information based on different spaces in the building. Next, a building walk will be organised to obtain an in-depth understanding of occupants' perceptions - mental, emotional and physical towards different spaces in the building. During the walk, researchers will engage participants in conversations about each space, and conversations will follow the materials experience framework to understand the relationship between materials, occupants and practises within the different spaces of the building \cite{giaccardi2015foundations}. Questions will also investigate the four different experiential levels (sensory, interpretive, affective and per formative) and will include asking about impressions of the space, usage of space \textit{(How would you use this space? Why do you use this space? What kind of tasks would you perform in this space?)}, emotions in the space \textit{(What emotions do you associate with this space?)}, comfort \textit{(Is this space comfortable? Is there sufficient light, warmth, ventilation?)}, shortcomings and scope for improvement. Participants will also be encouraged to take photos of artifacts that stand out to them. Data collected from both phases will be analysed using thematic analysis, and lexical analysis \cite{braun2006using, xue2020mood}. 

\subsection{Expected Contributions}
Through our case study, we aim to identify specific artifacts both tangible and intangible, and the properties of these artifacts that impact building occupants. Additionally, we expect to gain an understanding of the lived-in bodily comfort and emotional experiences within smart learning spaces from an affective, and interpretive perspective \cite{giaccardi2015foundations}. Our work is a preliminary step towards design and development of spaces that can understand and assist in the emotional and comfort needs of occupants help create experiences that occupants might find lacking in smart buildings (for eg., sense of autonomy and control in an automated building) \cite{moreno2014user}, or enhance certain other experiences (for eg., feeling of groundedness and familiarity in a space used by many) \cite{rehman2022personalisedcomfort}.

For example, the privacy curtains (Section \ref{subsec:building}, Figures \ref{fig:curtain} and \ref{fig:curtain-sheer}) could be better envisioned as a smart fabric that can intelligently transform from sheer to fully opaque to in-situ cater to varying privacy needs. Alternatively, materiality of objects could be expanded to materiality of space that in-situ infers occupants (negative) affective states, and attempts to alter it through means of emapthic responses. 





We believe that our findings may help put forth considerations for designing for occupants' comfort in built environments through the lens of material experiences design. 

Design of built spaces (and the artifacts they encompass) needs to be realised for the specific needs of occupants. Our findings may be used for the design and development of materials that can help create experiences that occupants might find lacking in smart hybrid learning spaces (for eg., sense of autonomy and control in an automated building) \cite{moreno2014user}, or enhance certain other experiences (for eg., feeling of groundedness and familiarity in a space used by many) \cite{rehman2022personalisedcomfort}.  For example, 
smart spaces could be designed with materials that in-situ infer occupants (negative) affective states, and attempt to alter it through means of emapthic responses. 

We ultimately envision a design philosophy for smart, hybrid learning spaces that allow materials to shape technology and the ``smartness" of the space. We wish to use our discussion to bring together research communities of material and interaction design with comfort studies and smart buildings.


\section{Conclusion}
The increased complexity of built environments in recent years can be attributed to a growing expectation for buildings to adapt to changing socio-environmental parameters. Given the inevitable growth of smart built environments, in this position paper we suggest that materials of a space can play a central role in understanding and shaping occupants comfort and emotions towards more human-centric smart buildings. We discuss why materiality (both tangible and intangible) of hybrid learning spaces needs to be examined and designed for occupants subjective affective and physical comfort. To that end, we highlight a case study that we will undertake to identify artifacts (tangible and intangible) and their specific properties that impact building occupants, and therefore needs rethinking and  redesign such that they may serve as catalysts for human comfort and mental health. 

\bibliographystyle{ACM-Reference-Format}
\bibliography{affective-comfort}

\end{document}
